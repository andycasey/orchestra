\documentclass{aastex61}

% For revision history
\IfFileExists{vc.tex}{\input{vc.tex}}{
    \newcommand{\githash}{UNKNOWN}
    \newcommand{\giturl}{UNKNOWN}}

% Define commands
\newcommand{\acronym}[1]{{\small{#1}}}
\newcommand{\project}[1]{\textsl{#1}}

\newcommand{\gaia}{\project{Gaia}}
\newcommand{\harps}{\acronym{HARPS}}

\received{}
\revised{}
\accepted{}

\submitjournal{AAS Journals}

\shorttitle{The Orchestra}
\shortauthors{Casey et al.}

\begin{document}

\title{An astrophysically-motivated data-driven model for stellar spectra}

\correspondingauthor{Andrew R. Casey}
\email{arc@ast.cam.ac.uk}


\author[0000-0003-0174-0564]{Andrew R. Casey}
\affil{Institute of Astronomy, Madingley Road, Cambridge CB3~0HA, England}
\affil{Monash Centre for Astrophysics, Monash University, Melbourne, 
	   VIC 3800, Australia}


\author{Melissa Ness}
\affil{Max-Planck-Institut f\"ur Astronomie, Heidelberg}


\author{David W. Hogg}
\affil{Center for Cosmology and Particle Physics, Department of Physics, 
	   New York University}
\affil{Center for Data Science, New York University}
\affil{Flatiron Institute, New York City}
\affil{Max-Planck-Institut f\"ur Astronomie, Heidelberg}

 
% Potential co-authors (contingent on significant contribution to this
% particular paper):
%\author{Megan Bedell}
%\author{Sven Buder}
%\author{Keith Hawkins}
%\author{Hans-Walter Rix}


\begin{abstract}
Abstract.
\end{abstract}

\keywords{}

\section{Introduction} 
\label{sec:introduction}
Write.


\section{Model}
\label{sec:model}


\begin{figure}
	\plotone{simple_solar_isochrone.pdf}
	\caption{
		Effective temperature $\teff$ and surface gravities $\logg$ from
		Solar metallicity \project{Dartmouth} isochrones \citep{dartmouth}
		with ages ${1, 2.5, 5, 7.5, 10, 12}$~Gyr. Effective temperatures
		and surface gravities at points along the isochrone were adopted
		to synthesize spectra.\label{fig:simple_solar_isochrone}}
\end{figure}


\begin{figure}
	\plotone{figures/simple_model/simple_photospheric_properties.pdf}
	\caption{
		Rosseland mean opacity $\kappa$ (top) and plasma temperature $T$ 
		(bottom) shown as a function of optical depth $\tau$ at 500~nm
		for Solar-abundance model photospheres with stellar parameters
		at the isochrone points shown in Figure \ref{fig:simple_solar_isochrone}.
		\label{fig:simple_photospheric_properties}}
\end{figure}


\begin{figure}
	\plotone{figures/simple_model/simple_regularization_performance.pdf}
	\caption{
		The root mean square deviation between a generalized linear model
		and the integrated equivalent width from spectral synthesis,
		shown as a function of increasing regularization strength $\Lambda$.
		We adopted $\Lambda = XX$ by $k$-fold ($k=20$) cross-validation.
		\label{fig:simple_regularization_performance}}
\end{figure}


\begin{figure}
	\plotone{figures/simple_model/simple_photospheric_coefficients.pdf}
	\caption{
		% TODO: Consistent with nomenclature?
		Optimized contribution function coefficients $\mathcal{C}$ as a 
		function of mean optical depth $\tau$ across all Solar metallicity 
		photospheres shown in Figure \ref{fig:simple_photospheric_properties}.
		Regularization has forced coefficients deep in the photosphere to
		zero, indicating that the atomic transition does not contribute net
		emission or absorption to the emergent spectrum.
		Positive coefficients indicate net emission and negative coefficients
		indicate net absorption.
		\label{fig:simple_photospheric_coefficients}}
\end{figure}


\begin{figure}
	\plotone{figures/simple_model/simple_photospheric_contributions.pdf}
	\caption{
		Contribution functions $\mathcal{C}\kappa/T$ as a function of mean
		optical depth $\tau$ across all Solar metallicity photospheres.
		Although there are non-zero contribution coefficients $\mathcal{C}$
		deep in the photosphere ($\log_{10}\tau < XX$; see Figure
		\ref{fig:simple_photospheric_coefficients}), this figure shows
		there is no significant contribution to the emergent spectrum from
		these deep layers.
		\label{fig:simple_photospheric_contributions}}
\end{figure}

			

\begin{figure}
	\plotone{figures/multi_model/multi_photospheric_coefficients.pdf}
	\caption{
		Contribution function coefficients $\mathcal{C}$ for models
		of different metallicities. These were modelled as a sum of two
		Gaussians (one emission, one absorption). The positions and
		strengths of these Gaussians can be seen as a function of [Fe/H].
		\label{fig:multi_photospheric_coefficients}}
\end{figure}


\begin{figure}
	\plotone{figures/multi_model/multi_predicted_ew.pdf}
	\caption{
		Predicted line strength (equivalent width) for the models
		shown in Figure \ref{fig:multi_photospheric_coefficients}.
		Points are coloured by their metallicity.
		The typical mean and root mean squared deviation is shown.
		Here we can see that the adopted model form can predict,
		for a single metallicity, the expected line strength
		across a star's entire lifetime.
		\label{fig:multi_predicted_ew}}
\end{figure}


\begin{figure}
	\plotone{figures/multi_model/multi_model_coefficients.pdf}
	\caption{
		Optimized model coefficients across many single-metallicity
		models.
		The behaviour of each coefficient is shown with respect to
		metallicity. This motivates the model form for our
		hierarchical model.
		\label{fig:multi_model_coefficients}}
\end{figure}


\acknowledgments

This project was developed in part at the 2016 NYC \gaia\ Sprint, hosted by the Center for Computational Astrophysics at the Simons Foundation in New York City.

This work has made use of data from the European Space Agency (ESA) mission \gaia\ (http://www.cosmos.esa.int/gaia), processed by the \gaia\ Data Processing and Analysis Consortium (DPAC, http://www.cosmos.esa.int/web/gaia/dpac/consortium). Funding for the DPAC has been provided by national institutions, in particular the institutions participating in the \gaia\ Multilateral Agreement.

% HARPS?

A.~R.~C. was supported in part by the European Union FP7 programme through ERC grant number 320360



\software{astropy \citep{Robitaille:2013}} 
% numpy, scipy

\begin{thebibliography}{}

\bibitem[Astropy Collaboration et al.(2013)]{Robitaille:2013} Astropy Collaboration, Robitaille, T.~P., Tollerud, E.~J., et al.\ 2013, \aap, 558, A33 

\end{thebibliography}


\end{document}
