\documentclass{aastex61}

% For revision history
\IfFileExists{vc.tex}{\input{vc.tex}}{
    \newcommand{\githash}{UNKNOWN}
    \newcommand{\giturl}{UNKNOWN}}

% Define commands
\newcommand{\acronym}[1]{{\small{#1}}}
\newcommand{\project}[1]{\textsl{#1}}
\newcommand{\tc}{\project{The Cannon}}
\newcommand{\thecannon}{\tc}

\newcommand{\gaia}{\project{Gaia}}
\newcommand{\harps}{\acronym{HARPS}}

\newcommand{\teff}{T_{\rm eff}}
\newcommand{\logg}{\log{g}}

\received{}
\revised{}
\accepted{}

\submitjournal{AAS Journals}

\shorttitle{The Orchestra}
\shortauthors{Casey, Hogg \& Ness}

\begin{document}

\title{An astrophysically-motivated data-driven model for stellar spectra}

\correspondingauthor{Andrew R. Casey}
\email{arc@ast.cam.ac.uk}


\author[0000-0003-0174-0564]{Andrew R. Casey}
\affil{Institute of Astronomy, Madingley Road, Cambridge CB3~0HA, England}
\affil{Monash Centre for Astrophysics, Monash University, Melbourne, 
	   VIC 3800, Australia}


\author{Melissa Ness}
\affil{Max-Planck-Institut f\"ur Astronomie, Heidelberg}


\author{David W. Hogg}
\affil{Center for Cosmology and Particle Physics, Department of Physics, 
	   New York University}
\affil{Center for Data Science, New York University}
\affil{Flatiron Institute, New York City}
\affil{Max-Planck-Institut f\"ur Astronomie, Heidelberg}



\author{others?}
 
% Potential co-authors (contingent on significant contribution to this
% particular paper):
%\author{Megan Bedell}
%\author{Sven Buder}
%\author{Keith Hawkins}
%\author{Hans-Walter Rix}


\begin{abstract}
We have shown that a quadratic data-driven model can predict stellar labels
more precisely than most physics-driven approaches.  This fact implies that 
a single quadratic function can accurately approximate the solutions to the 
coupled time-dependent magnetohydrodynamic and radiative equations, as well
as the level populations for every species in the photosphere.  Current 
applications of \thecannon\ treat all pixels independently, requiring up to
hundreds of coefficients per pixel to predict detailed chemical abundances.
Without priors on those coefficients and/or strong regularization, the model
interpretability can decrease as a consequence of the increased flexibility.
Stellar labels of peculiar stars may not be accurately predicted as a result,
making the model efficacy weakly dependent on both the training set and the 
chosen model form.  Here we derive a hierarchical linear model to approximate
spectral synthesis, based on state-of-the-art stellar photospheric models, 
line transition data, and modern spectral synthesis routines.  Our model 
predicts the contribution function for line transitions throughout a stellar 
photosphere, making it geometry invariant (1D or 3D), and capable of modelling
absorption or emission with or without the assumption of local thermodynamic
equilibrium.  We simultaneously infer stellar labels and model coefficients 
directly from data, thereby \emph{eliminating the requirement for a training 
set}.
We apply our model to \texttt{SOME\_DATA} and
\texttt{DO\_SOMETHING}.
We release software to approximate spectral syntheses at the 1\% level --- 
where systematics already dominate --- about $10^7$ times faster than modern
spectral synthesis programs.
\end{abstract}

\keywords{}

\section{Introduction} 
\label{sec:introduction}

% Data outgrown methods.
% Requirements for modern stellar spectroscopy.
% Tendency for models to become increasingly complex, hoping that will solve the common significant discpreancies noted between different choices of methods.
% (e.g., hoping that 3D models will save everyone because they don't require free parameters, which presumably is what is driving differences between spectroscopists)

% Data-driven models have shown that expensive spectral synthesis can be approximated by a simple linear or quadratic model.
% Not only does this approximation provide a 10^6 speed increase, it PERFORMS better.

% As one grows the parameters, the flexibility increases. Regularization showed that these models could be interpretable.
% However in some parts of parameter space, it would appear a single quadratic model is insufficient to explain everything.

% Worry about the training set.
% Most of these worries are reasonable, but people worry *too* much about this.
% Here we remove the requirement of a training set.



\section{Model}
\label{sec:model}


Here we will construct a linear model for the total relative absorption due to 
energy transition in a plasma. 

% Introduce E


absorption lines in stellar spectra.
The form of the model will be astrophysically-motivated, and the model coefficients
will be inferred from data. 


% Assumptions.





\begin{figure}
	\plotone{figures/simple_model/simple_solar_isochrone.pdf}
	\caption{
		Effective temperature $\teff$ and surface gravities $\logg$ from
		Solar metallicity \project{Dartmouth} isochrones \citep{dartmouth}
		with ages ${1, 2.5, 5, 7.5, 10, 12}$~Gyr. Effective temperatures
		and surface gravities at points along the isochrone were adopted
		to synthesize spectra.\label{fig:simple_solar_isochrone}}
\end{figure}


\begin{figure}
	\plotone{figures/simple_model/simple_photospheric_properties.pdf}
	\caption{
		Rosseland mean opacity $\kappa$ (top) and plasma temperature $T$ 
		(bottom) shown as a function of optical depth $\tau$ at 500~nm
		for Solar-abundance model photospheres with stellar parameters
		at the isochrone points shown in Figure \ref{fig:simple_solar_isochrone}.
		\label{fig:simple_photospheric_properties}}
\end{figure}


\begin{figure}
	\plotone{figures/simple_model/simple_regularization_performance.pdf}
	\caption{
		The root mean square deviation between a generalized linear model
		and the integrated equivalent width from spectral synthesis,
		shown as a function of increasing regularization strength $\Lambda$.
		We adopted $\Lambda = XX$ by $k$-fold ($k=20$) cross-validation.
		\label{fig:simple_regularization_performance}}
\end{figure}


\begin{figure}
	\plotone{figures/simple_model/simple_photospheric_coefficients.pdf}
	\caption{
		% TODO: Consistent with nomenclature?
		Optimized contribution function coefficients $\mathcal{C}$ as a 
		function of mean optical depth $\tau$ across all Solar metallicity 
		photospheres shown in Figure \ref{fig:simple_photospheric_properties}.
		Regularization has forced coefficients deep in the photosphere to
		zero, indicating that the atomic transition does not contribute net
		emission or absorption to the emergent spectrum.
		Positive coefficients indicate net emission and negative coefficients
		indicate net absorption.
		\label{fig:simple_photospheric_coefficients}}
\end{figure}


\begin{figure}
	\plotone{figures/simple_model/simple_photospheric_contributions.pdf}
	\caption{
		Contribution functions $\mathcal{C}\kappa/T$ as a function of mean
		optical depth $\tau$ across all Solar metallicity photospheres.
		Although there are non-zero contribution coefficients $\mathcal{C}$
		deep in the photosphere ($\log_{10}\tau < XX$; see Figure
		\ref{fig:simple_photospheric_coefficients}), this figure shows
		there is no significant contribution to the emergent spectrum from
		these deep layers.
		\label{fig:simple_photospheric_contributions}}
\end{figure}

			

\begin{figure}
	\plotone{figures/multi_model/multi_photospheric_coefficients.pdf}
	\caption{
		Contribution function coefficients $\mathcal{C}$ for models
		of different metallicities. These were modelled as a sum of two
		Gaussians (one emission, one absorption). The positions and
		strengths of these Gaussians can be seen as a function of [Fe/H].
		\label{fig:multi_photospheric_coefficients}}
\end{figure}


\begin{figure}
	\plotone{figures/multi_model/multi_predicted_ew.pdf}
	\caption{
		Predicted line strength (equivalent width) for the models
		shown in Figure \ref{fig:multi_photospheric_coefficients}.
		Points are coloured by their metallicity.
		The typical mean and root mean squared deviation is shown.
		Here we can see that the adopted model form can predict,
		for a single metallicity, the expected line strength
		across a star's entire lifetime.
		\label{fig:multi_predicted_ew}}
\end{figure}


\begin{figure}
	\plotone{figures/multi_model/multi_model_coefficients.pdf}
	\caption{
		Optimized model coefficients across many single-metallicity
		models.
		The behaviour of each coefficient is shown with respect to
		metallicity. This motivates the model form for our
		hierarchical model.
		\label{fig:multi_model_coefficients}}
\end{figure}


\section{Application to \harps\ spectra}
\label{sec:harps}





\acknowledgments

This project was developed in part at the 2016 NYC \gaia\ Sprint, hosted by the Center for Computational Astrophysics at the Simons Foundation in New York City.

This work has made use of data from the European Space Agency (ESA) mission \gaia\ 
(http://www.cosmos.esa.int/gaia), processed by the \gaia\ Data Processing and Analysis 
Consortium (DPAC, http://www.cosmos.esa.int/web/gaia/dpac/consortium). Funding for the
DPAC has been provided by national institutions, in particular the institutions 
participating in the \gaia\ Multilateral Agreement.
A.~R.~C. was supported in part by the European Union FP7 programme through ERC grant number 320360.
Based on observations made with ESO Telescopes at the La Silla Paranal Observatory under programme ID numbers XXXXXXXXXXXXXXXXX.




\software{astropy \citep{Robitaille:2013}, numpy, scipy, scikit-learn} 

\begin{thebibliography}{}

\bibitem[Astropy Collaboration et al.(2013)]{Robitaille:2013} Astropy Collaboration, Robitaille, T.~P., Tollerud, E.~J., et al.\ 2013, \aap, 558, A33 

\end{thebibliography}


\end{document}
